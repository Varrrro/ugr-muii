\section{Conclusiones}

Para facilitar la comparación de los modelos entrenados, se han recopilado en el
cuadro \ref{tab:test} los resultados obtenidos en las predicciones realizadas
sobre el conjunto de prueba que ya se han comentado a lo largo de la sección
anterior.

\begin{table}[h!]
    \caption{Resultados sobre el conjunto de prueba}
    \label{tab:test}
    \begin{center}
        \begin{tabular}{ |c|c| }
            \hline
            \textbf{Clasificador}  & \textbf{AUC} \\
            \hline
            \textit{Random Forest} & 0.59517723   \\
            \hline
            Perceptrón multicapa   & 0.52218163   \\
            \hline
            Regersión logística    & 0.5817315    \\
            \hline
        \end{tabular}
    \end{center}
\end{table}

El mejor modelo que hemos entrenado es, como se puede ver, el \textit{Random
    Forest}, con un AUC de 0.59517723, seguido de la regresión logística y con el
perceptrón multicapa en último lugar. El resultado del clasificador
\textit{Random Forest} podría ser incluso mejor usando un número de árboles
mayor a los que se han probado.

En general, los resultados son bastante malos, sin llegar ninguno de los modelos
siquiera al 0.6 de AUC y creo que esto se debe a que tan solo se usan 6
atributos de los más de 600 disponibles en el conjunto de datos inicial, así que
pienso que la calidad que es posible obtener está limitada.
