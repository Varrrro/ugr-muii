\section{Entrenamiento de los modelos}
Una vez procesados los datos y almacenados en la base de datos, se
pueden usar para entrenar los modelos, tanto ARIMA como el autorregresivo.
En el planteamiento inicial de la práctica, se proponía realizar este
entrenamiento dentro de los propios modelos, no obstante, este es un proceso
que consume mucho tiempo, por lo que he pensado que sería más eficiente
entrenar los modelos en tareas separadas del flujo de trabajo y almacenar
los modelos ya entrenados en ficheros, que luego se pasarían a los servicios
al lanzarlos.

\subsection{ARIMA}
Para entrenar el modelo ARIMA, utilizamos el código proporcionado, el cuál
hace uso de la librería \textit{pmdarima}. Debemos entrenar dos modelos, uno
para la temperatura y otro para la humedad. Una vez entrenados, usamos
los \textit{pickles} de Python para guardarlos en la carpeta \lstinline{/tmp/forecast/models/arima}.

\begin{itemize}
    \item\href{
        https://github.com/Varrrro/forecast/blob/master/airflow/tasks.py#L103-L123
    }{Operadores del flujo de trabajo}
    \item\href{
        https://github.com/Varrrro/forecast/blob/master/airflow/functions.py#L44-L92
    }{Entrenamiento de los modelos ARIMA}
\end{itemize}

\subsection{Modelo autorregresivo}
El modelo autorregresivo se entrena con la librería \textit{statsmodels},
que contiene gran cantidad de modelos estadísticos. El procedimiento es
análogo al entrenamiento de los modelos ARIMA, necesitando también entrenar
dos modelos y almacenándolos usando \textit{pickle}, esta vez en la carpeta
\lstinline{/tmp/forecast/models/autoreg}.

\begin{itemize}
    \item\href{
        https://github.com/Varrrro/forecast/blob/master/airflow/tasks.py#L125-L145
    }{Operadores del flujo de trabajo}
    \item\href{
        https://github.com/Varrrro/forecast/blob/master/airflow/functions.py#L94-L120
    }{Entrenamiento de los modelos autorregresivos}
\end{itemize}
