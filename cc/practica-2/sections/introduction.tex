\section{Introducción}
En esta práctica se pedía construir un servicio de predicción meteorológica
siguiendo el modelo conocido como \textbf{\textit{Cloud Native}}. Para ello,
se hace uso de la herramienta \textbf{Apache Airflow}\footnote{\href{https://airflow.apache.org/}{Web de Apache Airflow}},
la cuál permite definir flujos de trabajo con tareas atómicas que se pueden
ejecutar de forma distribuida, aprovechando las características de la nube.

En concreto, el servicio debe permitir obtener las predicciones de temperatura
y humedad para las próximas 24, 48 y 72 horas. Hay que implementar dos versiones
del mismo, usando en la primera un modelo de análisis de series temporales
llamado ARIMA. Para la segunda versión del servicio, he decidido usar un
modelo autorregresivo. Los dos servicios se ejecutarán usando contenedores.

Ambos modelos se entrenan con los conjuntos de datos proporcionados para la
práctica, los cuáles se deben procesar previamente en el flujo de trabajo
desarrollado.

A lo largo de este documento se presentarán las distintas tareas que se
debían realizar para la práctica, explicando la solución elegida e
incluyendo enlaces a la implementación de ésta, que se encuentra en GitHub\footnote{\href{https://github.com/varrrro/forecast}{Repositorio del proyecto}}.
