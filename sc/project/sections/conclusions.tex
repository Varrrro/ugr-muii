\section{Conclusiones}

En este proyecto se han realizado muchas pruebas y se ha obtenido una gran
cantidad de datos, de los cuáles se pueden extraer algunas conclusiones. La
primera que salta a la vista viendo los tiempos de respuesta de los contenedores
en ejecución, es que las aplicaciones contenerizadas pueden ser aptas para
restricciones de tiempo real. En esta línea, si volvemos a observar los datos
del cuadro \ref{tab:response-results}, la desviación típica en los tiempos en
i386 es considerablemente inferior a la que se aprecia en arm32v7 en cualquiera
de las situaciones (en ejecución, parado o pausado). En los tiempos de
lanzamiento también se aprecia ésto, aunque la diferencia de desviaciones es
menor, tal y como se ve en el cuadro \ref{tab:start-results}. Podríamos concluir
entonces que i386 ofrece un comportamiento más determinista que arm32v7, al
menos en los dispositivos usados.

Por otra parte, considero que los tiempos de lanzamiento que se han registrado
son demasiado elevados para sistemas con restricciones de tiempo real "duras",
aunque sí podrían ser aptos para los que tengan restricciones "blandas". El
principal problema lo encontramos en la poca consistencia de tiempos en todas
las tareas de gestión de los contenedores, tal y como se aprecia en los
resultados de los tiempos de respuesta para los contenedores parados y pausados
y de los tiempos de lanzamiento.

En definitiva, considero que los resultados son esperanzadores para la
aplicación de las tecnologías de contenerización a la automatización de la
industria, aunque creo que sería necesario construir herramientas específicas
para este caso de uso, como pueden ser motores y orquestadores de contenedores
que tengan en cuenta y garanticen los requisitos de tiempo de cada aplicación
contenerizada. Además, la plataforma de ARM necesita de mejoras en el campo del
trabajo con contenedores para poder ser una opción fiable.
