\section{Indexador}

Como su propio nombre indica, el indexador es la aplicación que construye el índice a partir de una colección de documentos dada. Se trata de una aplicación de línea de comandos que, como ya hemos indicado, hemos construido con \textbf{picocli}, una librería orientada a este tipo de aplicaciones. Esta librería nos permite definir \textit{flags} y otros parámetros de entrada para los comandos, facilitando su manejo. A continuación, se muestra la definición del comando creado para la indexación.

\begin{lstlisting}[language=Java]
@Command(name = "index", description = "Create or update an index from a set of documents.", mixinStandardHelpOptions = true, version = "index 1.0")
public class Index implements Callable<Integer> {

    @Parameters(paramLabel = "DIR", description = "Documents' directory to be indexed.", arity = "1..1")
    private File docsDir;

    @Option(names = { "-i", "--index" }, paramLabel = "INDEX", description = "Path to index. Defaults to /tmp/index.")
    private File indexFile = new File("/tmp/index");

    @Option(names = { "-s", "--stopwords" }, paramLabel = "STOPWORDS", description = "Stopwords file.", required = true)
    private File stopwordsFile;

    @Option(names = {"-u", "--update"}, description = "Update index instead of creating a new one.")
    private boolean update;

    public static void main(String... args) {
        int exitCode = new CommandLine(new Index()).execute(args);
        System.exit(exitCode);
    }
    
    @Override
    public Integer call() throws Exception {
        // Index documents
    }
}
\end{lstlisting}

Como se puede ver, nuestra aplicación requiere, por lo menos, de dos entradas: las rutas al fichero de palabras vacías (\textit{stopwords}) y a la colección de documentos. También se puede indicar una ruta para almacenar el índice aunque, si no se especifica, se usa \verb|/tmp/index| por defecto. Destacamos también el \textit{flag} \verb|-u|, el cuál permite indicar que se está actualizando un índice ya existente, con lo que no tenemos que realizar todo el proceso de nuevo si no es necesario.

En cuanto a la lógica que se encarga de realizar la indexación, es la siguiente:

\begin{lstlisting}[language=Java]
try {
    // Read stopwords from file
    var stopwords = readStopwords(stopwordsPath);

    // Set up index writer
    Directory dir = FSDirectory.open(indexPath);
    Analyzer analyzer = new SpanishAnalyzer(new CharArraySet(stopwords, true));
    IndexWriterConfig config = new IndexWriterConfig(analyzer);

    // Update index or create a new one
    if (update) {
        config.setOpenMode(OpenMode.CREATE_OR_APPEND);
    } else {
        config.setOpenMode(OpenMode.CREATE);
    }

    // Create index writer and process documents
    IndexWriter writer = new IndexWriter(dir, config);
    indexDocuments(writer, docsPath);

    // Close index writer
    writer.close();
} catch (Exception e) {
    System.out.println("Can't create index - Error: "+ e.getMessage());
    return 1;
}
\end{lstlisting}

El método \verb|indexDocuments| es el encargado de recorrer todos los archivos de la colección dada y llamar a \verb|indexDocument| para cada uno de ellos. Este último, extrae el texto del archivo PDF y añade el nuevo documento al índice, como se puede ver a continuación:

\begin{lstlisting}[language=Java]
public void indexDocument(IndexWriter writer, Path file) throws IOException {
    try (PDDocument pdf = PDDocument.load(file.toFile())) {
        // Extract text from PDF
        PDFTextStripper stripper = new PDFTextStripper();
        stripper.setLineSeparator("\n");
        stripper.setStartPage(1);
        String contents = stripper.getText(pdf);
        pdf.close();

        // Create Lucene document
        Document doc = new Document();

        // Add the path of the file
        doc.add(new StringField("path", file.toString(), Field.Store.YES));

        // Add the contents of the file
        doc.add(new TextField("contents", contents, Field.Store.NO));

        // Add the document to the index
        writer.addDocument(doc);
    }
}
\end{lstlisting}

Como se puede apreciar, indexamos tanto el contenido del documento como la ruta hacia el mismo.

\subsection{Manual de uso}

Usando el fichero \verb|jar| proporcionado, se puede ejecutar la aplicación de la siguiente manera:

\begin{lstlisting}
    java -jar indexer.jar --stopwords=[Ruta a fichero] --index=[Ruta a indice] [Ruta a coleccion]
\end{lstlisting}

Alternativamente, si se va a actualizar un índice ya creado, se puede usar:

\begin{lstlisting}
    java -jar indexer.jar -u --stopwords=[Ruta a fichero] --index=[Ruta a indice] [Ruta a coleccion]
\end{lstlisting}

\textbf{Nota:} Como ya se ha indicado anteriormente, no es completamente necesario indicar una ruta para el índice, usándose \verb|/tmp/index| por defecto.