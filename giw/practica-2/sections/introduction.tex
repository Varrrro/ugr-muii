\section{Introducción}

El objetivo de esta práctica es trabajar con una red social real para aplicar
las técnicas de análisis de este tipo de redes que se han estudiado en clase. En
concreto, la red social con la que se va a trabajar es Twitter, la cuál se
caracteriza especialmente por limitar el tamaño de las publicaciones de sus
usuarios a 280 caracteres.

Todo el trabajo de esta práctica se ha realizaco don la herramienta
\textit{Gephi} y un \textit{plugin} para la misma llamado \textit{Twitter
    streaming importer}, que es el que permite realizar la conexión con la red
social y "capturar" los \textit{tweets} e interacciones que van surgiendo. Este
es un detalle importante, ya que con este \textit{plugin} no podemos obtener
información sobre el pasado, si no que se captura la actividad en directo.

En concreto, he querido capturar y analizar la actividad relativa a las quejas
que se sucedieron en Twitter contra la Universidad de Granada, debido a los
problemas que surgieron con la adaptación a la enseñanza \textit{online} por la
crisis sanitaria del COVID-19. Con el análisis de la red obtenida, pretendo
encontrar a los usuarios más influyentes en este debate.
