\chapter{Conclusiones}
Una vez finalizado el proyecto, en esta sección se va a analizar el grado de
cumplimiento de los objetivos planteados al inicio del mismo y las sensaciones
que deja su desarrollo. Además, se discutirán posibles mejoras al sistema
planteado de cara a su futura extensión.

\section{Principales conclusiones}

Durante el desarrollo de este proyecto, se ha realizado una investigación
profunda y detallada sobre las tecnologías de virtualización y los conceptos más
importantes de la computación de tiempo real, buscando la manera de combinar
ambos aspectos para proporcionar una solución eficaz y práctica al problema
planteado por el IoT industrial. En este sentido, se ha estudiado la estructura
y funcionamiento de los sistemas operativos de tiempo real, incluyendo los
principales algoritmos de planificación de procesos con restricciones
temporales. También se ha analizado la situación de las soluciones basadas en
el kernel de Linux existentes en este campo, cumpliendo así con los primeros
objetivos planteados.

Estudiando las tecnologías de virtualización, planteamos que podía ser viable el
uso de contenedores para el despliegue sencillo de tareas de tiempo real en
múltiples dispositivos. Centrándonos en Docker, observamos que proporciona las
medidas de aislamiento entre procesos necesarias para la seguridad de los
sistemas, además de mecanismos y herramientas que facilitan el despliegue de sus
contenedores en plataformas de diversas arquitecturas. Al ser ejecutados por la
CPU como procesos normales, en teoría los procesos contenerizados podrían ser
planificados y controlados usando las políticas ya conocidas de planificación de
procesos en tiempo real. En el capítulo \ref{ch:03-performance_analysis} se
analizó el impacto que tiene el uso de contenedores sobre la latencia en la
activación de es tos procesos, concurriendo en que, aunque existente, este
impacto no es demasiado pronunciado y sigue siendo viable el uso de contenedores
para tareas con restricciones blandas, no para las duras. Con las pruebas
realizadas también nos dimos cuenta de que los mecanimos de gestión del ciclo de
vida de los contenedores que implementa Docker no ofrecen el comportamiento
determinista que necesitaríamos para incorporarlos a sistemas de tiempo real.
Por ello, aunque los procesos contenerizados sí son viables, las tareas de
despliegue los mismos se deben realizar de antemano.

En líneas generales, consideramos que los resultados apoyan claramente la
factibilidad del uso de contenedores para el despliegue de tareas de tiempo real
blandas. Con esto en mente, planteamos el diseño de una herramienta para
gestionar el despliegue de estas tareas sobre múltiples dispositivos de forma
centralizada y sencilla. El producto final que se ha implementado constituye una
prueba de concepto de lo que es posible realizar en este contexto, permitiendo
llevar a cabo las tareas deseadas y cumpliendo con los objetivos propuestos para
el mismo. Aún así, se trata de un software joven donde todavía se pueden
encontrar algunos errores y, sobre todo, para el que es necesario un mayor
período de prueba y validación antes de ser considerado su uso en entornos de
producción.

Esto, sumado a las buenas sensaciones obtenidas del uso inicial del sistema
desarrollado y a los resultados de la experimentación llevada a cabo, nos
permiten afirmar que el resultado del proyecto ha sido satisfactorio, aunque aún
queda trabajo por realizar para poder obtener una solución completamente pulida.

\section{Opinión personal}

Originalmente, este trabajo iba a estar más enfocado a una aplicación práctica
del uso de contenedores para el control de un dispositivo (p. ej., un brazo
robótico) de forma distribuida, aprovechando las características de los
contenedores para migrar los procesos de control de forma entre plataformas,
mejorando la fiabilidad y resiliencia del sistema. Debido a la situación
sanitaria, nos vimos obligados a cambiar el enfoque del trabajo hacia un
proyecto que fuera más factible de realizar en el contexto actual del
teletrabajo, pero manteniendo aún el deseo de ahondar en el uso de contenedores
para este tipo de aplicaciones. Así fue como llegamos a la idea del proyecto
final que se ha explicado a lo largo de este documento. A pesar de tener que
realizar este cambio, creo que las lecciones aprendidas durante su desarrollo
son muy valiosas, además de considerar que el sistema desarrollado sirve como
una clara prueba de la utilidad de las tecnologías de contenerización para la
implantación de los paradigmas del fog y edge computing.

Además de la situación extraordinaria en la que he tenido que desarrollar el
trabajo, los temas tratados en el mismo considero que son bastante complejos y,
en mi caso, completamente nuevos, lo que ha contribuido a su dificultad. No
obstante, considero que se trata de un conocimiento muy valioso que he obtenido
y es un área de conocimiento que me interesa y sobre la que quiero continuar
formándome.

El desarrollo de un sistema completo como este también me ha permitido
aplicar las técnicas y procedimientos de Ingeniería del Software que he
adquirido durante mi formación como ingeniero, lo cual considero muy valioso ya
que me proporciona experiencia en un proyecto real y completo. Mis conocimientos
de Python, Docker y sistemas operativos también se han ampliado y consolidado,
lo que considero que es muy positivo.

En general, estoy satisfecho tanto con el trabajo realizado como con los
resultados obtenidos. Creo firmemente que los contenedores son clave para la
implementación de modelos fog o edge en la industria, y pienso que el sistema
que he desarrollado es una buena prueba de concepto del tipo de herramientas que
serían necesarias para controlar esta clase de despliegues.

\section{Trabajo futuro}

En el marco del uso de contenedores para el despliegue de tareas de tiempo real,
se ha visto que es necesario invertir en el desarrollo de un motor especialmente
diseñado e implementado para este caso de uso, que asegure un comportamiento
determinista en sus operaciones y tenga en cuenta las particularidades de la
computación en tiempo real a la hora de ejecutar los contenedores. Estos
contenedores podrían usarse para combinar cargas de trabajo críticas y no
críticas, pero para ello es necesario, como ya indicaba Burns en
\cite{burns_mixed_2015}, disponer de métodos de análisis general para sistemas
de criticalidad mixta que tengan en cuenta todos los recursos del sistema.

En lo que respecta a la herramienta desarrollada en este trabajo, está claro que
se necesita hacer más hincapié en su validación, siendo necesario realizar más
pruebas con cargas de trabajo reales. También sería necesario añadir
procedimientos para gestionar el despliegue sobre dispositivos multiprocesador,
ya que actualmente esta situación no se maneja adecuadamente. El sistema
desarrollado solo puede ser pensado para el despliegue de tareas de tiempo real
blandas, ya que se ve limitado por el uso de Docker y su rendimiento. No
obstante, nada nos hace pensar que, si se consigue un motor de contenedores
especializado como el ya comentado, no se pueda usar también para tareas con
restricciones duras, aunque sería necesario llevar a cabo un proceso de
validación aún más intenso.

Desde el punto de vista de la usabilidad del sistema, sería muy beneficioso
contar con un cliente con interfaz gráfica para su gestión. Aunque el cliente de
línea de comandos que se ha implementado en este trabajo permite al usuario
acceder a todas las funcionalidades que ofrece el sistema actualmente, es cierto
que este tipo de aplicaciones no son tan sencillas de usar como, por ejemplo,
una aplicación web. Un panel de control gráfico facilita mucho a los usuarios el
control del sistema, ya que este se realiza de forma mucho más visual e
intuitiva que con la aplicación de línea de comandos ofrecida.

También hay algunas funcionalidades que consideramos que deberían añadirse al
sistema antes de poder ser usado en entornos reales, entre las que destaca
principalmente la capacidad de monitorizar el rendimiento de las tareas
desplegadas, controlando aspectos como el número de veces que se excede el
tiempo de ejecución definido o que se imcumple el límite de tiempo fijado. Esta
información es realmente útil para los usuarios del sistema, ya que les
permitiría comprobar el correcto funcionamiento del mismo y realizar ajustes
para su optimización. De la mano de esta funcionalidad, otra línea de trabajo
muy interesante sería la modificación dinámica de los atributos de planificación
de las tareas en base a los datos recogidos en su monitorización. Se podría
incorporar alguna heurística o, incluso, alguna solución más compleja de
aprendizaje automático que permitiera modificar tanto el tiempo de ejecución
como el límite de tiempo de una tarea en un nodo concreto en función de cómo
esté rindiendo o si está cumpliendo o no con las restricciones planteadas.