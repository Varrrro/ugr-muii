\chapter{Estimación de costes del proyecto}
\label{app:A-costs}

Al tratarse, mayoritariamente, de un proyecto de Ingeniería del Software, se ha
realizado también una estimación de los costes asociados a la realización del
mismo. Para ello, se ha aplicado un modelo basado en tiempo y materiales. En
2019, el coste salarial medio de un ingeniero informático recién salido de la
universidad en España era de entre 24.000 y 28.500 euros brutos al año.

En esta estimación, se ha asumido un salario mensual de 2.000€ brutos al mes,
que es aproximadamente un salario anual de 24.000€. Las tareas de coordinación y
dirección del jefe del proyecto se estiman en un 10\% del trabajo de ingeniería,
con un coste medio mensual de unos 5.000€ al mes para un ingeniero sénior.
Además, se ha supuesto también una jornada laboral que llega al máximo en España
de 40 horas semanales y que el porcentaje destinado a la Seguridad Social es del
30\%. También se tiene en cuenta el coste del hardware usado para pruebas. La
estimación final del coste de desarrollo de este proyecto se muestra en la tabla
\ref{tab:A-costs}.

\begin{table}[H]
    \centering
    \begin{tabular}{ | >{\columncolor[gray]{0.8}}l | p{0.2\textwidth} r | }
        \hline
        Raspberry Pi 4B 4GB         &  & 60,00€     \\
        \hline
        Cable Ethernet              &  & 7,00€      \\
        \hline
        Cable de alimentación USB-C &  & 7,00€      \\
        \hline
        Mano de obra ingeniero      &  & 10.400,00€ \\
        \hline
        Mano de obra ingeniero jefe &  & 2.600,00€  \\
        \hline
        \multicolumn{1}{ r |}{}     &  & 13.074,00€ \\
        \cline{2-3}
    \end{tabular}
    \caption{Desglose de costes del proyecto.}
    \label{tab:A-costs}
\end{table}

A todo esto habría que sumarle el coste de la estación de trabajo usada para el
desarrollo del proyecto, la cuál consiste de un ordenador de sobremesa o
portátil y los periféricos necesarios, junto con los gastos asociados al consumo
eléctrico y el acceso a internet. Estos gastos se han omitido de la estimación
realizada debido a que son muy variables y tampoco tienen un impacto muy
representativo en los costes del proyecto.

Cabe destacar que las cifras expuestas en este apéndice se corresponden
solamente con los costes del proyecto, no con su precio de mercado, ya que se
deber incluir también el porcentaje de beneficio a obtener por su realización.