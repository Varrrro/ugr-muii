\chapter{Instalación del kernel de Linux con \texttt{PREEMPT\_RT} en Raspberry Pi}
\label{app:B-rt_raspi}

Normalmente, para obtener un kernel de Linux parcheado con \texttt{PREEMPT\_RT}
es necesario descargarse por separado tanto la versión del kernel que queramos
usar como la versión del parche adecuada para dicho kernel. Luego, se aplica el
parche, se realizan algunos ajustes en la configuración para el dispositivo
específico en el que se va a usar y se compila el nuevo kernel. Se trata de un
proceso relativamente sencillo, pero que requiere de conocimientos avanzados
sobre el kernel de Linux. Por suerte, la comunidad de usuarios de Raspberry Pi
mantiene  de forma pública versiones del kernel parcheadas con
\texttt{PREEMPT\_RT} para estos dispositivos, lo que facilita enormemente el
proceso de instalación. En este apéndice, se describe el proceso a seguir para
instalar en una Raspberry Pi 4B una de estas versiones del kernel de Linux
parcheado para tiempo real.

Cabe destacar que, aunque la Raspberry Pi es de arquitectura ARM, el kernel se
puede preparar en un ordenador independiente que sea x86, que es lo que se ha
hecho para este trabajo y lo que se explica en este apéndice. En este caso, lo
que se realiza es una compilación cruzada.

Antes de comenzar, es necesario crear los directorios sobre los que se va a
trabajar y clonar los repositorios de GitHub donde se guardan tanto las
versiones del kernel como las herramientas necesarias para su compilación. En el
caso del repositorio que contiene los kernels, clonamos la rama de la versión
concreta que queremos usar. Las ramas que contienen versiones parcheadas para
tiempo real se identifican con el sufijo \texttt{-rt}. Tal y como se puede ver
en el listado \ref{lst:B-workdir_prep}, hemos usado la versión 4.19 del kernel,
dado que es la más reciente que se encuentra disponible con el parche en el
momento en el que se ha realizado el trabajo.

\begin{lstlisting}[
    language=bash,
    caption={Preparación de las carpetas y descarga de los repositorios.},
    label={lst:B-workdir_prep},
    morekeywords={mkdir}
]
mkdir -p %\mytilde)/rpi-rt/rt-kernel && cd ~/rpi-rt
git clone https://github.com/raspberrypi/linux.git 
    -b rpi-4.19.y-rt
git clone https://github.com/raspberrypi/tools.git
\end{lstlisting}

Antes de compilar, en el listado \ref{lst:B-env_vars} se muestra como se guarda
en variables de entorno algunos valores que serán necesarios para realizar esta
compilación.

\begin{lstlisting}[
    language=bash,
    caption={Definición de variables de entorno.},
    label={lst:B-env_vars}
]
export ARCH=arm
export CROSS_COMPILE=%\mytilde)/rpi-rt/tools/arm-bcm2708/
    gcc-linaro-arm-linux-gnueabihf-raspbian-x64/bin/
    arm-linux-gnueabihf-
export INSTALL_MOD_PATH=%\mytilde)/rpi-rt/rt-kernel
export INSTALL_DTBS_PATH=%\mytilde)/rpi-rt/rt-kernel
export KERNEL=kernel7l
\end{lstlisting}

Una vez hecho esto, procedemos con la compilación. Para ello, se deben ejecutar
los objetivos del \texttt{Makefile} ubicado en \lstinline{\mytilde/rpi-rt/linux}
y que se muestran en el listado \ref{lst:B-kernel_compilation}. En este listado,
el valor \texttt{X} que se le pasa a la opción \texttt{-j} debe ser igual al
número de núcleos de CPU que posea el ordenador en el que se está realizando la
compilación, con el objetivo de que se haga más rápido.

\begin{lstlisting}[
    language=bash,
    caption={Compilación del kernel parcheado.},
    label={lst:B-kernel_compilation},
    morekeywords={make}
]
make bcm2711_defconfig
make -jX zImage
make -jX modules
make -jX dtbs
make -jX modules_install
make -jX dtbs_install
\end{lstlisting}

Cuando termina la compilación, debemos añadir a la nueva imagen la información
relativa al proceso de inicio del dispositivo o \textit{boot}, ejecutando los
comandos que se muestran en el listado \ref{lst:B-boot_info}.

\begin{lstlisting}[
    language=bash,
    caption={Añadir información de \textit{boot}.},
    label={lst:B-boot_info},
    morekeywords={mkdir, mv}
]
mkdir $INSTALL_MOD_PATH/boot
./scripts/mkknlimg ./arch/arm/boot/zImage
        $INSTALL_MOD_PATH/boot/$KERNEL.img
cd $INSTALL_MOD_PATH/boot
mv $KERNEL.img rt_kernel.img
\end{lstlisting}

Todos los archivos de la carpeta \lstinline{\mytilde/rpi-rt/rt-kernel} se
comprimen entonces en un \texttt{tar.gz} que se copia a la Raspberry Pi, a la
cuál debemos acceder ahora para ejecutar los comandos del listado
\ref{lst:B-kernel_install}, que son los que consiguen instalar el nuevo kernel.

\begin{lstlisting}[
    language=bash,
    caption={Instalación del kernel.},
    label={lst:B-kernel_install}
]
tar -xzf rt-kernel.tar.gz
cd boot && sudo cp -rd * /boot/
cd ../lib && sudo cp -rd * /lib/
cd ../overlays && sudo cp -d * /boot/overlays
cd .. && sudo cp -d bcm* /boot/
\end{lstlisting}

Para terminar, añadimos \lstinline{kernel=rt_kernel.img} al fichero
\texttt{/boot/config} y reiniciamos la Raspberry Pi. Tras el reinicio, el
comando \lstinline{uname -r} nos tiene que indicar la nueva versión del kernel,
que, en nuestro caso, es \texttt{4.19.71-rt24-v7l+}. Ahora, el kernel puede
realizar planificación apropiativa, además de recibir cambios en otros
mecanismos que reducen su latencia, con lo que la Raspberry Pi puede ejecutar
tareas de tiempo real.
