\chapter{Ejecución de las pruebas de \texttt{rt-tests} dentro de contenedores}
\label{app:C-container_tests}

Como se ha visto en el capítulo \ref{ch:03-performance_analysis}, para estudiar
el impacto que tiene el uso de contenedores sobre la latencia en la activación
de procesos de tiempo real se ha hecho uso de la utilidad
\texttt{cyclicdeadline} de la suite de pruebas de tiempo real para Linux
\texttt{rt-tests}. Para obtener los resultados deseados, era necesario ejecutar
estas pruebas tanto de forma «nativa» como dentro de contenedores. En este
apéndice se expone la imagen Docker que se ha preparado para ejecutar no solo
\texttt{cyclicdeadline}, sino cualquier prueba de \texttt{rt-tests} dentro de un
contenedor\footnote{El contenedor solo ha sido probado con \texttt{cyclictest},
    \texttt{cyclicdeadline} y \texttt{deadline\_test}, aunque debería funcionar
    correctamente para el resto de utilidades también.}. En el listado
\ref{lst:C-dockerfile} se puede ver el \texttt{Dockerfile} con esta definición.

\begin{lstlisting}[
    language=Dockerfile,
    caption={\texttt{Dockerfile} que define una imagen para la ejecución de \texttt{rt-tests}.},
    label={lst:C-dockerfile}
]
FROM debian:stable-slim

RUN apt update && apt install -y \
        git \
        build-essential \
        bison \
        flex \
        libnuma-dev \
        python3 \
        python3-distutils

RUN git clone \
    git://git.kernel.org/pub/scm/utils/rt-tests/rt-tests.git
WORKDIR ./rt-tests

RUN git checkout tags/v1.8 -b v1.8
RUN make all && make install

VOLUME [ "/sys/kernel/debug" ]

ENTRYPOINT [ "/bin/sh", "-c" ]
\end{lstlisting}

La imagen base escogida es la edición liviana o \textit{slim} de la versión más
reciente de Debian disponible. En este caso, realmente no importaba demasiado
cuál elegir, tan solo se necesitan las utilidades mínimas. A continuación, se
deben instalar las dependencias necesarias, entre las que figuran \texttt{git},
la librería de NUMA (\textit{Non Uniform Memory Access}) usada por
\texttt{rt-tests} y los paquetes necesarios para la construcción del código. Una
vez hecho esto, se clona el repositorio de \texttt{rt-tests} y se cambia el
directorio de trabajo a la nueva carpeta. Las distintas versiones de la suite se
encuentran identificadas en el repositorio por \textit{tags} o etiquetas, por lo
que obtenemos la versión deseada descargando los \textit{commits} de su etiqueta
en una nueva rama. En nuestro caso, y como se puede apreciar en el listado,
estamos usando la versión 1.8, siendo la 1.9 la más reciente en el momento en el
que se realizó el trabajo. El motivo por el que se ha escogido esta versión en
vez de la 1.9 es sencillamente que esta última provoca fallos en la compilación
para arquitecturas ARM debido a la introducción de una nueva dependencia. Una
vez tenemos el código descargado, construimos e instalamos las librerías y los
ejecutables con \texttt{make}.

Se puede apreciar como, a continuación de esta compilación, se declara un
volumen para el contenedor. En Docker, los volúmenes son mecanismos que permiten
vincular un directorio del sistema de ficheros del anfitrión con otro del
sistema de ficheros del contenedor, de forma que su contenido sea accesible
desde dentro del mismo. Esto es necesario para la carpeta
\texttt{/sys/kernel/debug}, ya que \texttt{rt-tests} lee varios de los ficheros
que contiene, especialmente \texttt{sched\_features}, para determinar las
propiedades del sistema.

Para terminar, se define \lstinline{/bin/sh -c} como el punto de entrada del
contenedor. Esto quiere decir que, al ejecutarlo, se lanzará este comando y se
pasarán como argumentos los que se indiquen en el \lstinline{docker run}. De
esta forma, tenemos la posibilidad de elegir la prueba de \texttt{rt-tests} que
se desea ejecutar en cada caso.

\begin{lstlisting}[
    language=bash,
    caption={Ejemplo de ejecución de \texttt{cyclicdeadline} usando el contenedor.},
    label={lst:C-docker_run},
    showstringspaces=false
]
docker run
    -v /sys/kernel/debug:/sys/kernel/debug
    --cap-add SYS_NICE --cap-add IPC_LOCK
    rt-tests "cyclicdeadline -D 10m -t 4"
\end{lstlisting}

En el listado \ref{lst:C-docker_run}, se muestra un ejemplo de ejecución de
\texttt{cyclicdeadline} usando la imagen de contenedor definida anteriormente.
Como se puede apreciar, es necesario usar varios argumentos en la ejecución de
\texttt{docker run} para hacer que el contenedor se lance correctamente. Estos
argumentos son:

\begin{itemize}
    \item \texttt{-v}: Se trata de la conexión del volumen definido dentro del
          contenedor y que hemos explicado antes, con un directorio del sistema de
          ficheros del anfitrión. Aunque no es necesario usar el mismo directorio
          tanto en el anfitrión como en el contenedor, en este caso sí que lo es
          debido ya que los ficheros deben estar dentro de ese directorio específico.
    \item \texttt{--cap-add}: Con este argumento, podemos indicar algunas
          \textit{capabilities} o capacidades de Linux que queremos que tenga el
          contenedor. Estas capacidades permiten realizar diversas operaciones.
          Para nuestro contenedor, necesitamos dos: \texttt{SYS\_NICE}, para
          asignar prioridades a los procesos (valor \textit{nice}); e
          \texttt{IPC\_LOCK}, para usar las funciones de reserva de memoria (p.
          ej., \texttt{mlock} o \texttt{mlockall}).
\end{itemize}

Después del nombre de la imagen a ejecutar, que en nuestro caso es
\texttt{rt-tests}, se indica entre comillas la cadena de texto que se le pasará
al punto de entrada del contenedor, es decir, a \lstinline{/bin/sh -c}. En el
ejemplo del listado, se ejecuta la prueba \texttt{cyclicdeadline} con una
duración de 10 minutos y un total de 4 hilos.
