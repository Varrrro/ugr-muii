\pdfbookmark[1]{Resumen}{Resumen}

\chapter*{Resumen}

\selectlanguage{spanish}
Con el advenimiento de la Industria 4.0, las empresas buscan incorporar técnicas
de inteligencia artificial y análisis de datos a sus instalaciones y procesos
industriales con el objetivo de mejorar la productividad y la autonomía. En este
trabajo, se plantea la posibilidad de usar tecnologías de contenerización de
procesos para el despliegue eficiente de estas nuevas tareas junto con las de
tiempo real habituales en los sistemas de control industrial, estudiando las
distintas tecnologías posibles y realizando un análisis del rendimiento de
tareas de tiempo real contenerizadas con Docker. Además, se diseña e implementa
una herramienta software que sirve como prueba de concepto para el despliegue y
la orquestación de este tipo de tareas sobre entornos distribuidos mediante el
uso de contenedores.

\selectlanguage{english}
\textit{
    As Industry 4.0 gets closer, companies are looking to incorporate artificial
    intelligence and data analysis techniques to their facilities and industrial
    processes with the objective of improving productivity and autonomy. This paper
    proposes the use of processes containerization techonologies for efficient
    deployment of these new tasks along with the real-time tasks that are common in
    industrial control systems, studying different possible techonologies and
    analysing the performance of real-time tasks containerized using Docker.
    Finally, a proof-of-concept software tool for deployment and orchestration of
    this type of tasks on distributed environments by means of containers.
}

\selectlanguage{spanish}

\cleardoublepage