\pdfbookmark[1]{Resumen}{Resumen}

\chapter*{Resumen}

\selectlanguage{spanish}
Con el advenimiento de la Industria 4.0, las empresas buscan incorporar técnicas
de inteligencia artificial y análisis de datos a sus instalaciones y procesos
industriales con el objetivo de mejorar la productividad y la autonomía. En este
trabajo, se explora la aplicación de los paradigmas de computación distribuida
en la nube y el \textit{fog computing} como modelos para la implementación de
los sistemas de control del futuro. Dentro de este marco, se plantea la
posibilidad de usar tecnologías de contenerización de procesos para el
despliegue eficiente y flexible de estos nuevos sistemas de control industrial,
estudiando las distintas tecnologías posibles y realizando un análisis del
rendimiento de tareas de tiempo real con restricciones blandas contenerizadas
con Docker para comprobar su viabilidad. Además, se diseña e implementa una
herramienta software que sirve como prueba de concepto para el despliegue y la
orquestación de este tipo de tareas sobre entornos distribuidos mediante el uso
de contenedores.

\selectlanguage{english}
\textit{As Industry 4.0 gets closer, companies are looking to incorporate
    artificial intelligence and data analysis techniques to their facilities and
    industrial processes with the objective of improving productivity and
    autonomy. This paper explores the use of distributed computing in the cloud
    and fog as models for the implementation of the control systems of the
    future. Along with this, the use of processes containerization technologies
    is proposed as a way of achieving effiecient and flexible deployments of
    these new industrial control systems, studying different possible
    techonologies and analysing the performance of soft real-time tasks
    containerized using Docker to check its viability. Finally, a
    proof-of-concept software tool for deployment and orchestration of this type
    of tasks on distributed environments by means of containers.}

\selectlanguage{spanish}

\cleardoublepage