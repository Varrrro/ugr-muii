\section{Introducción}
El objetivo de esta práctica es obtener un modelo de red neuronal capaz de clasificar las imágenes de \href{http://yann.lecun.com/exdb/mnist/}{la base de datos de dígitos manuscritos de MNIST}. Se trata de imágenes de 28x28 píxeles que representan números del 0 al 9 escritos a mano.

Para ello, he implementado y experimentado con varios modelos hasta obtener el mejor resultado. Todas las implementaciones se han realizado con el lenguaje Go y haciendo uso del \href{https://godoc.org/gonum.org/v1/gonum/mat}{paquete \textit{mat} de la librería Gonum} para realizar el álgebra lineal con matrices necesario.

La base de datos de MNIST nos proporciona 60000 ejemplos para entrenamiento y 10000 para la prueba de nuestros modelos. A lo largo de este documento, se van a exponer las distintas implementaciones que se han realizado y los resultados obtenidos al clasificar estos dos conjuntos para cada una de ellas.